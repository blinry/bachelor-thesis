\documentclass[twoside]{scrreprt}

\usepackage[T1]{fontenc}
\usepackage[utf8]{inputenc}

\usepackage{microtype}

\title{The Nut Shell}
\subtitle{A Framework for Creating Interactive Command Line Tutorials}
\author{Sebastian Morr}
\publishers{Betreuer: Dr. Werner Struckmann}
\date{\today}

\begin{document}
\maketitle
\tableofcontents

\begin{abstract}
        (Purpose)

        (Design + evaluation methods)

        (Major results)

        (Summary of conclusions)
\end{abstract}

\chapter{Introduction}

(Topic: Command line tutorials)

(Motivation: Traditional, static tutorials have problems)

    (Attention shift)

    (No goal affirmation)

    (No reaction to mistakes)

(Core idea: Interlace tutorial text and CLI)

(Role model: Text adventures)

(Research question: Is this approach "better"?)

(Prior approaches)

    (Try Ruby/Git/Haskell)

    (What's missing in them)

(Conventions in this thesis)

\chapter{Overview}

        (Goals and principles)

            (Basic event loop: prompt -> editing -> execution/output)

            (Adaptability to arbitrary "targets")

            (Annotation + environment changing, otherwise raw CLI interaction)

        (Steps of this thesis)

        (Diagram: Layers)

\chapter{Design}

\section{Command line parser}

\subsection{Purpose}

                (Recognizing parts of command line output)

\subsection{Background}

                (How a terminal works)

                (Escape characters)

                (Readline key combinations)

\subsection{Architecture}

                (Component diagram)

                (Parser EBNF)

\subsection{Problems and workarounds}

                (Command line editing)

                (...)

\section{Internal DSL}

            (Purpose: High-level layer around CLI parser)

            (Description of necessary DSL calls)

\section{The nutsh language}

\subsection{Purpose}

                (Describes a self-contained teaching unit, a "lesson")

\subsection{Design goals}

                (Easy to read and write)

                    (Use syntax the user already knows: Regex + Go syntax)

                    (Introduce new syntax for often-used semantics)

                (Minimize redundance)

                    (DRY, allow reuse of code snippets)

\subsection{Properties}

                (String-based)

                (Functional)

\subsection{Lexical elements}

                (Token types)

                (Diagram: State machine)

\subsection{Syntax}

                (EBNF of language constructs)

\subsection{Parsing}

                (How YACC works)

\subsection{Semantics}

                (Specification of language constructs)

\subsection{Interpreter}

                (State)

                (Function stack)

\subsection{Automated testing}

                (Motivation)

                (Testing algorithm)

\subsection{Examples}

\chapter{Implementation}

\section{Used technologies}

            (Go)

            (kr/pty for terminal emulation)

\section{High-level design}

            (Diagram: Package diagram)

\section{Command line tool}

\subsection{Usage}

\subsection{Builtin functions}

\chapter{Application and evaluation}

\section{Methods}

\subsection{Setting}

                (Description of setting: Preparatory course for CS students)

                (Previous teaching method)

                (Groups)

\subsection{Tutorial}

                (Content, examples)

                (Best practises in lesson writing)

\subsection{Survey}

                (Questions)

\section{Results}

            (Pretty graphs)

            (Statistical evaluation)

\chapter{Conclusions}

        (Discussion of survey results)

\chapter{Limitations and future directions}

        (Future directions:)

            (Automated typo detection)

            (Simplified prompt syntax)

            (Lesson dependency tree)

        (Outlook, future of the Nut Shell)

\chapter{Acknowledgements}

        (...)

\chapter{References}

        (...)

\appendix

\chapter{Example lesson}

        (nutsh source code)

        (Execution output)

\chapter{Table of terminal escape codes}

        (...)

\chapter*{Affidavit}

        ("Erklärung an Eides statt")

\end{document}
